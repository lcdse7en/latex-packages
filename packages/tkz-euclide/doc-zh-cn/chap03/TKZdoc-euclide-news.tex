\documentclass[../main.tex]{subfiles}
\begin{document}
% \section{News and compatibility}
\section{新特性与兼容性}

% Some changes have been made to make the syntax more homogeneous and especially
% to distinguish the definition and search for coordinates from the rest, i.e.
% drawing, marking and labelling.
% In the future, the definition macros being isolated, it will be easier to
% introduce a phase of coordinate calculations using \tkzimp{Lua}.
宏包更新后,增强了语法一致性。
尤其是将点的定义与点的计算分离后,为后续绘图、标记和标注操作带来了方便。
在未来,将更好地分离宏定义,并引入\tkzimp{Lua},以方便坐标计算。

% An important novelty is the recent replacement of the \tkzNamePack{fp} package
% by \tkzNamePack{xfp}.  This is to improve the calculations a little bit more and
% to make it easier to use.

一个重要的新特性是使用\tkzNamePack{xfp}宏包替换了\tkzNamePack{fp}宏包,从而进一步提升了计算性能,
并使得用户使用更为便捷。

% Here are some of the changes.
以下是主要的更新:
\vspace{1cm}
\begin{itemize}\setlength{\itemsep}{10pt}

% \item Improved code and bug fixes;
\item 优化了代码并修复了部分Bug;

% \item With \tkzimp{tkz-euclide} loads all objects, so there's no need to place
% \tkzcname{usetkzobj\{all\}};\item The bounding box is now controlled in each
% macro (hopefully) to avoid the use of \tkzcname{tkzInit} followed by
% \tkzcname{tkzClip};\item Added macros for the bounding box: \tkzcname{tkzSaveBB}
% \tkzcname{tkzClipBB} and so on;\item  Logically most macros accept \TIKZ\
% options. So I removed the \enquote{duplicate} options when possible thus the \enquote{label
% options} option is removed;
\item 由于\tkzimp{tkz-euclide}默认载入了所有对象,因此无需再使用
\tkzcname{usetkzobj\{all\}}进行载入操作;

\item 在各个命令中单独处理包围盒,从而避免使用\tkzcname{tkzInit}命令和\tkzcname{tkzClip}命令

\item 增加了\tkzcname{tkzSaveBB}、\tkzcname{tkzClipBB}等包围盒处理命令;
\item 逻辑上,由于所有命令都可以直接使用\TIKZ{}选项,所以尽可能移除了如
\enquote{label options}等\enquote{重复}选项;

% \item Random points are now in \tkzname{\tkznameofpack} and the macro
% \tkzcname{tkzGetRandPointOn} is replaced by the macro \tkzcname{tkzDefRandPointOn}. For
% homogeneity reasons, the points must be retrieved with \tkzcname{tkzGetPoint};
\item 用\tkzname{\tkznameofpack}宏包生成随机点,并且使用\tkzcname{tkzDefRandPointOn}命令代替了
\tkzcname{tkzGetRandPointOn}命令,
并要求必须使用\tkzcname{tkzGetPoint}引用生成的随机点;

% \item The options \tkzname{end} and \tkzname{start} which allowed to give a
% label to a straight  line are removed. You now have to use the macro
% \tkzcname{tkzLabelLine};
\item 新增\tkzcname{tkzLabelLine}命令,用于为直线添加标注。
删除了原来的直线标注选项\tkzname{end}和\tkzname{start}。

% \item Introduction of the libraries \NameLib{quotes} and \NameLib{angles}; it
% allows to give a label to a point, even if I am not in favour of this practice;
\item 新增用于为点添加标注的\NameLib{quotes}和\NameLib{angles}库。

% \item The notion of vector disappears, to draw a vector just pass \enquote{->} as an
% option to \tkzcname{tkzDrawSegment};
\item 删除了向量操作,为\tkzcname{tkzDrawSegment}命令添加了\enquote{->}以绘制向量;

% \item Many macros still exist, but are obsolete and will disappear:
\item 仍然存在但已过时并将要删除的命令有:
\begin{itemize}

% \item |\tkzDrawMedians| trace and create midpoints on the sides of a triangle.
% The creation and drawing separation is not respected so it is preferable to
% first create the coordinates of these points with |\tkzSpcTriangle[median]|
% and then draw with |\tkzDrawSegments| or |\tkzDrawLines|;
\item 用于跟踪和创建三角形各边中点的命令|\tkzDrawMedians|,不符合计算和绘制分离的思想,
因此,建议首先使用 |\tkzSpcTriangle[median]|命令计算中点,然后使用|\tkzDrawSegments|或|\tkzDrawLines|绘制。

% \item |\tkzDrawMedians(A,B)(C)| is now spelled |\tkzDrawMedians(A,C,B)|. This
% defines the median from $C$;
\item 将|\tkzDrawMedians(A,B)(C)|命令改为|\tkzDrawMedians(A,C,B)|,
以定义中点$C$;

% \item Another example |\tkzDrawTriangle[equilateral]| was handy but it is better
% to get the third point with |\tkzDefTriangle[equilateral]| and then draw with
% |\tkzDrawPolygon|;
\item |\tkzDrawTriangle[equilateral]|命令将被删除,
建议先用|\tkzDefTriangle[equilateral]|命令定义点,
然后用|\tkzDrawPolygon|绘制多边形;

% \item |\tkzDefRandPointOn| is replaced by |\tkzGetRandPointOn|;\item now
% |\tkzTangent| is replaced by |\tkzDefTangent|;
\item 用|\tkzGetRandPointOn|命令替换了|\tkzDefRandPointOn|命令;
\item 用|\tkzDefTangent|命令替换了|\tkzTangent|命令;

% \item You can use |global path name| if you want find intersection  but it's
% very slow like in \TIKZ.
\item 可以使用|global path name|参数求交点,但像在\TIKZ{}中一样,其计算速度非常慢。

\end{itemize}

% \item Appearance of the macro \tkzcname{usetkztool} which allows to load new
% \enquote{tools}.
\item 引入了\tkzcname{usetkztool}命令,以加载新的\enquote{tools}库。
\end{itemize}

\end{document}
\endinput
