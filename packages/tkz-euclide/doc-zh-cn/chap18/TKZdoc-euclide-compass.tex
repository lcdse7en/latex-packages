\documentclass[../main.tex]{subfiles}
\begin{document}
% \section{Using the compass}
\section{尺规标记}

% \subsection{Main macro \tkzcname{tkzCompass}}
\subsection{\tkzcname{tkzCompass}命令:绘制尺规标记}

% \begin{NewMacroBox}{tkzCompass}{\oarg{local options}\parg{A,B}}%
% This macro allows you to leave a compass trace, i.e. an arc at a designated
% point. The center must be indicated. Several specific options will modify the
% appearance of the arc as well as TikZ options such as style, color, line
% thickness etc.
%
% You can define the length of the arc with the option |length| or the option
% |delta|.
%
% \medskip
% \begin{tabular}{lll}%
% \toprule
% options             & default & definition                        \\
% \midrule
% \TOline{delta} {0 (deg)}{Modifies the angle of the arc by increasing it
% symmetrically (in degrees)}
% \TOline{length}{1 (cm)}{Changes the length (in cm)}
% \end{tabular}
% \end{NewMacroBox}
\begin{NewMacroBox}{tkzCompass}{\oarg{命令选项}\parg{A,B}}%
该命令绘制尺规标记,即一小段圆弧。
使用该命令时,须指定圆心。
可以使用\TIKZ{}的style、color、line thickness等样式
设置标记外观。

可以使用|length|或|delta|选项指定标记长度。

\medskip
\begin{tabular}{lll}%
\toprule
选项             & 默认值 & 含义                        \\
\midrule
\TOline{delta} {0 (deg)}{延伸长度(度)}
\TOline{length}{1 (cm)}{圆弧长度(cm)}
\end{tabular}
\end{NewMacroBox}

% \subsubsection{Option \tkzname{length}}
\subsubsection{\tkzname{length}选项示例}

\begin{tkzexample}[latex=7cm,small]
\begin{tikzpicture}
  \tkzDefPoint(1,1){A}
  \tkzDefPoint(6,1){B}
  \tkzInterCC[R](A,4cm)(B,3cm)
  \tkzGetPoints{C}{D}
  \tkzDrawPoint(C)
  \tkzCompass[color=red,length=1.5](A,C)
  \tkzCompass[color=red](B,C)
  \tkzDrawSegments(A,B A,C B,C)
\end{tikzpicture}
\end{tkzexample}

% \subsubsection{Option \tkzname{delta}}
\subsubsection{\tkzname{delta}选项示例}

\begin{tkzexample}[latex=7cm,small]
\begin{tikzpicture}
  \tkzDefPoint(0,0){A}
  \tkzDefPoint(5,0){B}
  \tkzInterCC[R](A,4cm)(B,3cm)
  \tkzGetPoints{C}{D}
  \tkzDrawPoints(A,B,C)
  \tkzCompass[color=red,delta=20](A,C)
  \tkzCompass[color=red,delta=20](B,C)
  \tkzDrawPolygon(A,B,C)
  \tkzMarkAngle(A,C,B)
\end{tikzpicture}
\end{tkzexample}

% \subsection{Multiple constructions \tkzcname{tkzCompasss}}
\subsection{\tkzcname{tkzCompasss}命令:绘制多个尺规标记}

% \begin{NewMacroBox}{tkzCompasss}{\oarg{local options}\parg{pt1,pt2, pt3,pt4,\dots}}%
% \tkzHandBomb{}Attention the arguments are lists of two points. This saves a few
% lines of code.
%
% \medskip
% \begin{tabular}{lll}%
% \toprule
% options             & default & definition                        \\
% \midrule
% \TOline{delta} {0}{Modifies the angle of the arc by increasing it symmetrically}
% \TOline{length}{1}{Changes the length}
% \end{tabular}
% \end{NewMacroBox}
\begin{NewMacroBox}{tkzCompasss}{\oarg{命令选项}\parg{pt1,pt2, pt3,pt4,\dots}}%
\tkzHandBomb{}注意:参数是点对列表。

\medskip
\begin{tabular}{lll}%
\toprule
选项             & 默认值 & 含义                        \\
\midrule
\TOline{delta} {0}{延伸角度}
\TOline{length}{1}{圆弧长度}
\end{tabular}
\end{NewMacroBox}

\begin{tkzexample}[latex=8cm,small]
\begin{tikzpicture}[scale=.75]
  \tkzDefPoint(2,2){A}
  \tkzDefPoint(5,-2){B}
  \tkzDefPoint(3,4){C}
  \tkzDrawPoints(A,B)
  \tkzDrawPoint[color=red,shape=cross out](C)
  \tkzCompasss[color=orange](A,B A,C B,C C,B)
  \tkzShowLine[mediator,color=red,
                      dashed,length = 2](A,B)
  \tkzShowLine[parallel = through C,
                    color=blue,length=2](A,B)
  \tkzDefLine[mediator](A,B)
  \tkzGetPoints{i}{j}
  \tkzDefLine[parallel=through C](A,B)
  \tkzGetPoint{D}
  \tkzDrawLines[add=.6 and .6](C,D A,C B,D)
  \tkzDrawLines(i,j) \tkzDrawPoints(A,B,C,i,j,D)
  \tkzLabelPoints(A,B,C,i,j,D)
\end{tikzpicture}
\end{tkzexample}

% \subsection{Configuration macro \tkzcname{tkzSetUpCompass}}
\subsection{\tkzcname{tkzSetUpCompass}命令:设置尺规标记样式}

% \begin{NewMacroBox}{tkzSetUpCompass}{\oarg{local options}}%
% \begin{tabular}{lll}%
% options             & default & definition                        \\
% \midrule
% \TOline{line width}  {0.4pt}{line thickness}
% \TOline{color}  {black!50}{line colour}
% \TOline{style}  {solid}{solid line style, dashed,dotted, \dots}
% \end{tabular}
% \end{NewMacroBox}
\begin{NewMacroBox}{tkzSetUpCompass}{\oarg{命令选项}}%
\begin{tabular}{lll}%
选项            & 默认值 & 含义                        \\
\midrule
\TOline{line width}  {0.4pt}{线宽}
\TOline{color}  {black!50}{颜色}
\TOline{style}  {solid}{线型:solid、dashed、dotted、\dots}
\end{tabular}
\end{NewMacroBox}

% \subsubsection{Use of \tkzcname{tkzSetUpCompass}}
\subsubsection{示例}

\begin{tkzexample}[latex=7cm,small]
\begin{tikzpicture}[showbi/.style={bisector,
                    size=2,gap=3}, scale=.75]
  \tkzSetUpCompass[color=blue,line width=.3 pt]
  \tkzDefPoints{0/1/A, 8/3/B, 3/6/C}
  \tkzDrawPolygon(A,B,C)
  \tkzDefLine[bisector](B,A,C) \tkzGetPoint{a}
  \tkzDefLine[bisector](C,B,A) \tkzGetPoint{b}
  \tkzShowLine[showbi](B,A,C)
  \tkzShowLine[showbi](C,B,A)
  \tkzInterLL(A,a)(B,b) \tkzGetPoint{I}
  \tkzDefPointBy[projection= onto A--B](I)
  \tkzGetPoint{H}
  \tkzDrawCircle[radius,color=gray](I,H)
  \tkzDrawSegments[color=gray!50](I,H)
  \tkzDrawLines[add=0 and -.2,color=blue!50](A,a B,b)
  \tkzShowBB
\end{tikzpicture}
\end{tkzexample}

\end{document}
\endinput
