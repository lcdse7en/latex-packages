\documentclass[../main.tex]{subfiles}
\begin{document}
% \section{Protractor}
\section{量角器}

% Based on an idea by Yves Combe, the following macro allows you to draw a
% protractor.
% The operating principle is even simpler. Just name a half-line (a ray). The
% protractor will be placed on the origin $O$, the direction of the half-line is
% given by $A$. The angle is measured in the direct direction of the trigonometric
% circle.
基于\tkzimp{Yves Combe}的方法,
其工作原理更为简单,仅半条直线(射线),
量角器原点位于点$O$,射线方向由$A$确定。
角度方向由指定的测量圆方向决定。
\subsection{\tkzcname{tkzProtractor}命令:绘制量角器}
% \begin{NewMacroBox}{tkzProtractor}{\oarg{local options}\parg{$O,A$}}%
% \begin{tabular}{lll}%
% options    & default & definition     \\
% \midrule
% \TOline{lw}  {0.4 pt} {line thickness}
% \TOline{scale}  {1} {ratio: adjusts the size of the protractor}
% \TOline{return} {false} {trigonometric circle indirect}
% \end{tabular}
% \end{NewMacroBox}
\begin{NewMacroBox}{tkzProtractor}{\oarg{命令选项}\parg{$O,A$}}%
\begin{tabular}{lll}%
选项    & 默认值 & 含义     \\
\midrule
\TOline{lw}  {0.4 pt} {线宽}
\TOline{scale}  {1} {比例: 用于调整量角器尺寸}
\TOline{return} {false} {反向测量圆}
\end{tabular}
\end{NewMacroBox}

% \subsection{The circular protractor}
\subsection{正向圆量角器}

% Measuring in the forward direction
正向测量圆方向

\begin{tkzexample}[latex=7cm,small]
\begin{tikzpicture}[scale=.5]
  \tkzDefPoint(2,0){A}\tkzDefPoint(0,0){O}
  \tkzDefShiftPoint[A](31:5){B}
  \tkzDefShiftPoint[A](158:5){C}
  \tkzDrawPoints(A,B,C)
  \tkzDrawSegments[color = red,
    line width = 1pt](A,B A,C)
  \tkzProtractor[scale = 1](A,B)
\end{tikzpicture}
\end{tkzexample}

% \subsection{The circular protractor, transparent and returned}
\subsection{反向圆量角器}
逆向测量圆方向

\begin{tkzexample}[latex=7cm,small]
\begin{tikzpicture}[scale=.5]
  \tkzDefPoint(2,3){A}
  \tkzDefShiftPoint[A](31:5){B}
  \tkzDefShiftPoint[A](158:5){C}
  \tkzDrawSegments[color=red,line width=1pt](A,B A,C)
  \tkzProtractor[return](A,C)
\end{tikzpicture}
\end{tkzexample}

\end{document}
\endinput
