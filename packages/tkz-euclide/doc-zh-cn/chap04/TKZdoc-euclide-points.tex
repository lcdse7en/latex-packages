\documentclass[../main.tex]{subfiles}
\begin{document}
% \section{Definition of a point}
\section{定义点}

% Points can be specified in any of the following ways:
% \begin{itemize}
% \item Cartesian coordinates;
% \item Polar coordinates;
% \item Named points;
% \item Relative points.
% \end{itemize}
可以通过如下方式定义一个点:
\begin{itemize}
\item 笛卡尔坐标;
\item 极坐标;
\item 为点命名;
\item 相对位置.
\end{itemize}

% Even if it's possible, I think it's a bad idea to work directly with
% coordinates. Preferable is to use named points.
% A point is defined if it has a name linked to a unique pair of decimal numbers.
% Let $(x,y)$ or $(a:d)$  i.e. ($x$ abscissa, $y$ ordinate) or ($a$ angle: $d$
% distance).
虽然仅用坐标就可以定义一个点,但建议为点进行命名。
如果一个名称唯一地与一个十进制数对应,则可作为该点的名称,
由于正交笛卡尔坐标系定义了一个平面,其坐标轴是正交的,
因此,可以用$(x,y)$或$(a:d)$表示该平面中的一个点,其中$x$是横坐标,$y$是纵坐标;$a$是角度,$d$是距离,
% This is possible because the plan has been provided with an orthonormed
% Cartesian coordinate system. The working axes are supposed to be (ortho)normed
% with unity equal to $1$~cm or something equivalent like $0.39370$~in.
% Now by default if you use a grid or axes, the rectangle used is defined by the
% coordinate points: $(0,0)$ and $(10,10)$. It's the macro \tkzcname{tkzInit} of
% the package \tkzNamePack{tkz-base} that creates this rectangle. Look at the
% following two codes and the result of their compilation:
其度量单位等于$1$~cm或$0.39370$~in。

现在,默认情况下,如果使用网格或坐标轴,则可由$(0,0)$ 和$(10,10)$两个坐标定义一个矩形。

可以用\tkzNamePack{tkz-base}宏包的\tkzcname{tkzInit}命令创建该矩形,
请参阅以下两个代码及其绘制结果,以体会这些命令的作用。

\begin{tkzexample}[latex=10.5cm,small]
\begin{tikzpicture}
  \tkzGrid
  \tkzDefPoint(0,0){O}
  \tkzDrawPoint[red](O)
  \tkzShowBB[line width=2pt,teal]
\end{tikzpicture}
\end{tkzexample}

\begin{tkzexample}[latex=7cm,small]
\begin{tikzpicture}
  \tkzDefPoint(0,0){O}
  \tkzDefPoint(5,5){A}
  \tkzDrawSegment[blue](O,A)
  \tkzDrawPoints[red](O,A)
  \tkzShowBB[line width=2pt,teal]
\end{tikzpicture}
\end{tkzexample}

% The Cartesian coordinate $(a,b)$ refers to the point $a$ centimeters in the
% $x$-direction and $b$ centimeters in the $y$-direction.
笛卡尔坐标$(a,b)$表示该点的$x$坐标是$a$~cm,$y$坐标是$b$~cm。

% A point in polar coordinates requires an angle $\alpha$, in degrees, and a
% distance  $d$ from the origin with a dimensional unit by default it's the
% \texttt{cm}.
极坐标中表示一个点则需要一个角度$\alpha$(度)和一个从原点度量的距离$d$(默认单位是\texttt{cm})。

% Cartesian coordinates\par
笛卡尔坐标\par
\begin{tkzexample}[latex=7cm,small]
\begin{tikzpicture}[scale=1]
  \tkzInit[xmax=5,ymax=5]
  \tkzDefPoints{0/0/O,1/0/I,0/1/J}
  \tkzDrawXY[noticks,>=latex]
  \tkzDefPoint(3,4){A}
  \tkzDrawPoints(O,A)
  \tkzLabelPoint(A){$A_1 (x_1,y_1)$}
  \tkzShowPointCoord[xlabel=$x_1$,
                     ylabel=$y_1$](A)
  \tkzLabelPoints(O,I)
  \tkzLabelPoints[left](J)
  \tkzDrawPoints[shape=cross](I,J)
\end{tikzpicture}
\end{tkzexample}

% Polar coordinates\par
极坐标\par
\begin{tkzexample}[latex=7cm,small]
\begin{tikzpicture}[scale=1]
  \tkzInit[xmax=5,ymax=5]
  \tkzDefPoints{0/0/O,1/0/I,0/1/J}
  \tkzDefPoint(40:4){P}
  \tkzDrawXY[noticks,>=triangle 45]
  \tkzDrawSegment[dim={$d$,
                  16pt,above=6pt}](O,P)
  \tkzDrawPoints(O,P)
  \tkzMarkAngle[mark=none,->](I,O,P)
  \tkzFillAngle[fill=blue!20,
                opacity=.5](I,O,P)
  \tkzLabelAngle[pos=1.25](I,O,P){$\alpha$}
  \tkzLabelPoint(P){$P (\alpha : d )$}
  \tkzDrawPoints[shape=cross](I,J)
  \tkzLabelPoints(O,I)
  \tkzLabelPoints[left](J)
\end{tikzpicture}
\end{tkzexample}

% The \tkzNameMacro{tkzDefPoint} macro is used to define a point by assigning
% coordinates to it. This macro is based on \tkzNameMacro{coordinate}, a macro of
% \TIKZ. It can use \TIKZ-specific options such as \tkzname{shift}. If
% calculations are required then the \tkzNamePack{xfp} package is chosen. We can
% use Cartesian or polar coordinates.
可以用\tkzNameMacro{tkzDefPoint}命令,通过指定坐标定义一个点。
该命令基于\TIKZ{}的\tkzNameMacro{coordinate}命令实现,因此,可在该命令中使用类似\tkzname{shift}
的\TIKZ{}的选项,
该命令使用\tkzNamePack{xfp}宏包实现计算。在定义点时,既可以使用笛卡尔坐标,也可以使用极坐标。

% \subsection{Defining a named point  \tkzcname{tkzDefPoint}}
\subsection{\tkzcname{tkzDefPoint}命令:定义命名点}

% \begin{NewMacroBox}{tkzDefPoint}{\oarg{local options}\parg{$x,y$}\marg{name} or \parg{$\alpha$:$d$}\marg{name}}%
% \begin{tabular}{lll}%
% arguments &  default & definition  \\
% \midrule
% \TAline{($x,y$)}{no default}{$x$ and $y$ are two dimensions, by default in cm.}
% \TAline{($\alpha$:$d$)}{no default}{$\alpha$ is an angle in degrees, $d$ is a dimension}
% \TAline{\{name\}}{no default}{Name assigned to the point: $A$, $T_a$ ,$P1$, \dots}
% \bottomrule
% \end{tabular}
%
% \medskip
% The obligatory arguments of this macro are two dimensions expressed with
% decimals, in the first case they are two measures of length, in the second case
% they are a measure of length and the measure of an angle in degrees.
%
% \medskip
% \begin{tabular}{lll}%
% \toprule
% options             & default & definition  \\
% \midrule
% \TOline{label} {no default} {allows you to place a label at a predefined
% distance}
% \TOline{shift} {no default} {adds $(x,y)$ or $(\alpha:d)$ to all coordinates}
% \end{tabular}
% \end{NewMacroBox}
\begin{NewMacroBox}{tkzDefPoint}{\oarg{命令选项}\parg{$x,y$}\marg{名称} or \parg{$\alpha$:$d$}\marg{名称}}%
\begin{tabular}{lll}%
参数 &  默认值 & 含义  \\
\midrule
\TAline{($x,y$)}{无}{$x$ 和 $y$ 分别是2维坐标,默认单位是cm.}
\TAline{($\alpha$:$d$)}{无}{$\alpha$是角度(度), $d$是距离(cm)}
\TAline{\{名称\}}{无}{点的名称,如:$A$, $T_a$ ,$P1$, \dots}
\bottomrule
\end{tabular}

\medskip
必选参数是十进制表示的2维坐标值,
笛卡尔坐标表示两个长度,极坐标表示角度和距离。

\medskip
\begin{tabular}{lll}%
\toprule
选项             & 默认值 & 含义  \\
\midrule
\TOline{label} {无} {按预设的距离添加标注}
\TOline{shift} {无} {为$(x,y)$或$(\alpha:d)$添加坐标偏移}
\end{tabular}
\end{NewMacroBox}

% \subsubsection{Cartesian coordinates }
\subsubsection{笛卡尔坐标示例}

\begin{tkzexample}[latex=7cm,small]
\begin{tikzpicture}
  \tkzInit[xmax=5,ymax=5]
  \tkzDefPoint(0,0){A}
  \tkzDefPoint(4,0){B}
  \tkzDefPoint(0,3){C}
  \tkzDrawPolygon(A,B,C)
  \tkzDrawPoints(A,B,C)
\end{tikzpicture}
\end{tkzexample}

% \subsubsection{Calculations with \tkzNamePack{xfp}}
\subsubsection{使用\tkzNamePack{xfp}宏包实现计算}

\begin{tkzexample}[latex=7cm,small]
\begin{tikzpicture}[scale=1]
  \tkzInit[xmax=4,ymax=4]
  \tkzGrid
  \tkzDefPoint(-1+2,sqrt(4)){O}
  \tkzDefPoint({3*ln(exp(1))},{exp(1)}){A}
  \tkzDefPoint({4*sin(pi/6)},{4*cos(pi/6)}){B}
  \tkzDrawPoints[color=blue](O,B,A)
\end{tikzpicture}
\end{tkzexample}

% \subsubsection{Polar coordinates}
\subsubsection{极坐标示例}

\begin{tkzexample}[vbox,small]
  \begin{tikzpicture}
  \foreach \an [count=\i] in {0,60,...,300}
    {\tkzDefPoint(\an:3){A_\i}}
  \tkzDrawPolygon(A_1,A_...,A_6)
  \tkzDrawPoints(A_1,A_...,A_6)
  \end{tikzpicture}
\end{tkzexample}

\newpage

% \subsubsection{Calculations and coordinates}
\subsubsection{坐标计算}

% You must follow the syntax of \tkzNamePack{xfp} here. It is always possible to
% go through \tkzNamePack{pgfmath} but in this case, the coordinates must be
% calculated before using the macro \tkzcname{tkzDefPoint}.
在计算坐标时,需遵循\tkzNamePack{xfp}宏包语法。
另外,如使用\tkzNamePack{pgfmath}库计算,
则必须在使用\tkzcname{tkzDefPoint}命令前完成计算。

\begin{tkzexample}[latex=6cm,small]
  \begin{tikzpicture}[scale=.5]
  \foreach \an [count=\i] in {0,2,...,358}
    {\tkzDefPoint(\an:sqrt(sqrt(\an mm))){A_\i}}
  \tkzDrawPoints(A_1,A_...,A_180)
  \end{tikzpicture}
\end{tkzexample}

% \subsubsection{Relative points}
\subsubsection{相对位置点}

% First, we can use the \tkzNameEnv{scope} environment from \TIKZ.
% In the following example, we have a way to define an equilateral triangle.
可以使用相对位置定义一个点,此时,需使用\TIKZ{}的\tkzNameEnv{scope}环境。

下面的代码给出了一种定义等边三角形的方法:

\begin{tkzexample}[latex=7cm,small]
\begin{tikzpicture}[scale=1]
  \tkzSetUpLine[color=blue!60]
  \begin{scope}[rotate=30]
    \tkzDefPoint(2,3){A}
    \begin{scope}[shift=(A)]
       \tkzDefPoint(90:5){B}
       \tkzDefPoint(30:5){C}
    \end{scope}
  \end{scope}
  \tkzDrawPolygon(A,B,C)
  \tkzLabelPoints[above](B,C)
  \tkzLabelPoints[below](A)
  \tkzDrawPoints(A,B,C)
\end{tikzpicture}
\end{tkzexample}

%<--------------------------------------------------------------------------->
% \subsection{Point relative to another: \tkzcname{tkzDefShiftPoint}}
\subsection{\tkzcname{tkzDefShiftPoint}命令:定义平移点}
% \begin{NewMacroBox}{tkzDefShiftPoint}{\oarg{Point}\parg{$x,y$}\marg{name} or \parg{$\alpha$:$d$}\marg{name}}%
% \begin{tabular}{lll}%
% arguments &  default & definition \\
% \midrule
% \TAline{($x,y$)}{no default}{$x$ and $y$ are two dimensions, by default in cm.}
% \TAline{($\alpha$:$d$)}{no default}{$\alpha$ is an angle in degrees, $d$ is a dimension}
%
% \midrule
% options &  default & definition \\
%
% \midrule
% \TOline{[pt]} {no default} {\tkzcname{tkzDefShiftPoint}[A](0:4)\{B\}}
% \end{tabular}
% \end{NewMacroBox}
\begin{NewMacroBox}{tkzDefShiftPoint}{\oarg{参考点}\parg{$x,y$}\marg{名称} 或 \parg{$\alpha$:$d$}\marg{名称}}%
\begin{tabular}{lll}%
参数 &  默认值 & 含义 \\
\midrule
\TAline{($x,y$)}{无}{$x$ 和 $y$是2维坐标,默认单位是cm.}
\TAline{($\alpha$:$d$)}{无}{$\alpha$是角度(度),$d$是距离}

\midrule
选项 &  默认值 & 含义 \\

\midrule
\TOline{[参考点]} {无} {例如:\tkzcname{tkzDefShiftPoint}[A](0:4)\{B\}}
\end{tabular}
\end{NewMacroBox}

\newpage

% \subsubsection{Isosceles triangle with  \tkzcname{tkzDefShiftPoint}}
\subsubsection{用\tkzcname{tkzDefShiftPoint}命令构造等腰三角形}

% This macro allows you to place one point relative to another. This is equivalent
% to a translation. Here is how to construct an isosceles triangle with main
% vertex $A$ and angle at vertex of $30^{\circ}$.
% This macro allows you to place one point relative to another. This is equivalent
该命令允许基于一个点定义另一个点,等价于点的平移。
下面的代码给出了一种通过点$A$和$30^{\circ}$顶角等腰三角形的构造方法。

\begin{tkzexample}[latex=7cm,small]
\begin{tikzpicture}[rotate=-30]
  \tkzDefPoint(2,3){A}
  \tkzDefShiftPoint[A](0:4){B}
  \tkzDefShiftPoint[A](30:4){C}
  \tkzDrawSegments(A,B B,C C,A)
  \tkzMarkSegments[mark=|,color=red](A,B A,C)
  \tkzDrawPoints(A,B,C)
  \tkzLabelPoints(B,C)
  \tkzLabelPoints[above left](A)
\end{tikzpicture}
\end{tkzexample}

% \subsubsection{Equilateral triangle}
\subsubsection{用\tkzcname{tkzDefShiftPoint}命令构造等边三角形}

% Let's see how to get an equilateral triangle (there is much simpler)
下面的代码给出了一种极为简捷的等边三角形构造方法。

\begin{tkzexample}[latex=7cm,small]
\begin{tikzpicture}[scale=1]
  \tkzDefPoint(2,3){A}
  \tkzDefShiftPoint[A](30:3){B}
  \tkzDefShiftPoint[A](-30:3){C}
  \tkzDrawPolygon(A,B,C)
  \tkzDrawPoints(A,B,C)
  \tkzLabelPoints(B,C)
  \tkzLabelPoints[above left](A)
  \tkzMarkSegments[mark=|,color=red](A,B A,C B,C)
\end{tikzpicture}
\end{tkzexample}

% \subsubsection{Parallelogram}
\subsubsection{用\tkzcname{tkzDefShiftPoint}命令构造平行四边形}

% There's a simpler way
简单的定义平行四边形的方式为:

\begin{tkzexample}[latex=7cm,small]
\begin{tikzpicture}
  \tkzDefPoint(0,0){A}
  \tkzDefPoint(30:3){B}
  \tkzDefShiftPointCoord[B](10:2){C}
  \tkzDefShiftPointCoord[A](10:2){D}
  \tkzDrawPolygon(A,...,D)
  \tkzDrawPoints(A,...,D)
\end{tikzpicture}
\end{tkzexample}

%<--------------------------------------------------------------------------->
% \subsection{Definition of multiple points: \tkzcname{tkzDefPoints}}
\subsection{\tkzcname{tkzDefPoints}命令:定义点集}

% \begin{NewMacroBox}{tkzDefPoints}{\oarg{local options}\marg{$x_1/y_1/n_1,x_2/y_2/n_2$,\dots}}%
% $x_i$ and $y_i$ are the coordinates of a referenced point $n_i$
\begin{NewMacroBox}{tkzDefPoints}{\oarg{命令选项}\marg{$x_1/y_1/n_1,x_2/y_2/n_2$,\dots}}%
$x_i$ 和 $y_i$ 是$n_i$点的2维坐标

\begin{tabular}{lll}%
\toprule
参数 &  默认值  & 样例  \\
\midrule
\TAline{$x_i/y_i/n_i$}{无}{\tkzcname{tkzDefPoints\{0/0/O,2/2/A\}}}
\end{tabular}

\medskip
\begin{tabular}{lll}%
选项             & 默认值 & 含义   \\
\midrule
\TOline{shift} {无} {为所有点添加$(x,y)$或$(\alpha:d)$坐标偏移}
\end{tabular}
\end{NewMacroBox}

% \subsection{Create a triangle}
\subsubsection{用\tkzcname{tkzDefPoints}命令构造三角形}
\begin{tkzexample}[latex=6cm,small]
\begin{tikzpicture}[scale=1]
  \tkzDefPoints{0/0/A,4/0/B,4/3/C}
  \tkzDrawPolygon(A,B,C)
  \tkzDrawPoints(A,B,C)
\end{tikzpicture}
\end{tkzexample}

% \subsection{Create a square}
\subsubsection{用\tkzcname{tkzDefPoints}命令构造正方形}

% Note here the syntax for drawing the polygon.
注意该代码中绘制多边形的语法。
\begin{tkzexample}[latex=6cm,small]
\begin{tikzpicture}[scale=1]
  \tkzDefPoints{0/0/A,2/0/B,2/2/C,0/2/D}
  \tkzDrawPolygon(A,...,D)
  \tkzDrawPoints(A,B,C,D)
\end{tikzpicture}
\end{tkzexample}

% \section{Special points}
\section{定义特殊点}

% The introduction of the dots was done in \tkzname{tkz-base}, the most important
% macro being \tkzcname{tkzDefPoint}. Here are some special points.
以上是\tkzname{tkz-base}宏包中基本的定义一个点的命令,其中,最为重要的是\tkzcname{tkzDefPoint}
命令。当然,该宏包也提供了方便地直接定义一些特殊点的命令。
%<--------------------------------------------------------------------------->
% \subsection{Middle of a segment \tkzcname{tkzDefMidPoint}}
\subsection{\tkzcname{tkzDefMidPoint}命令:定义线段中点}

% It is a question of determining the middle of a segment.
% \begin{NewMacroBox}{tkzDefMidPoint}{\parg{pt1,pt2}}%
% The result is in \tkzname{tkzPointResult}. We can access it with
% \tkzcname{tkzGetPoint}.
%
% \medskip
%
% \begin{tabular}{lll}%
% \toprule
% arguments & default & definition \\
% \midrule
% \TAline{(pt1,pt2)}{no default}{pt1 and pt2 are two points}
% \end{tabular}
% \end{NewMacroBox}
\begin{NewMacroBox}{tkzDefMidPoint}{\parg{pt1,pt2}}%
定义的点存储于\tkzcname{tkzPointResult}命令中,
也可以通过\tkzcname{tkzGetPoint}命令得到该点,并为其命名。

\medskip

\begin{tabular}{lll}%
\toprule
参数 & 默认值 & 含义 \\
\midrule
\TAline{(pt1,pt2)}{无}{pt1和pt2是线段的两个端点}
\end{tabular}
\end{NewMacroBox}

% \subsubsection{Use of \tkzcname{tkzDefMidPoint}}
\subsubsection{\tkzcname{tkzDefMidPoint}命令示例}

% Review the use of \tkzcname{tkzDefPoint} in \tkzNamePack{tkz-base}.

\begin{tkzexample}[latex=6cm,small]
\begin{tikzpicture}[scale=1]
  \tkzDefPoint(2,3){A}
  \tkzDefPoint(4,0){B}
  \tkzDefMidPoint(A,B) \tkzGetPoint{C}
  \tkzDrawSegment(A,B)
  \tkzDrawPoints(A,B,C)
  \tkzLabelPoints[right](A,B,C)
\end{tikzpicture}
\end{tkzexample}

% \subsection{Barycentric coordinates }
\subsection{重心坐标}

% $pt_1$, $pt_2$, \dots, $pt_n$ being $n$ points, they define $n$ vectors
% $\overrightarrow{v_1}$, $\overrightarrow{v_2}$, \dots, $\overrightarrow{v_n}$
% with the origin of the referential as the common endpoint. $\alpha_1$,
% $\alpha_2$,\dots $\alpha_n$ are $n$ numbers, the vector obtained by:
设共有$pt_1$, $pt_2$, \dots, $pt_n$  $n$ 个点,则它们定义了$n$个向量
$\overrightarrow{v_1}$, $\overrightarrow{v_2}$, \dots, $\overrightarrow{v_n}$。
令$\alpha_1$,$\alpha_2$,\dots $\alpha_n$是$n$常数, 因此可按下式得到一个新向量:

\begin{align*}
  \frac{\alpha_1 \overrightarrow{v_1} + \alpha_2 \overrightarrow{v_2} + \cdots +
  \alpha_n \overrightarrow{v_n}}{\alpha_1
  + \alpha_2 + \cdots + \alpha_n}
\end{align*}

% defines a single point.
则,该向量可用于定义一个点。

% \begin{NewMacroBox}{tkzDefBarycentricPoint}{\parg{pt1=$\alpha_1$,pt2=$\alpha_2$,\dots}}%
% \begin{tabular}{lll}%
% arguments & default & definition \\
% \midrule
% \TAline{(pt1=$\alpha_1$,pt2=$\alpha_2$,\dots)}{no default}{Each point has a
% assigned weight}
% \bottomrule
% \end{tabular}
%
% \medskip
% You need at least two points.
% \end{NewMacroBox}
\begin{NewMacroBox}{tkzDefBarycentricPoint}{\parg{pt1=$\alpha_1$,pt2=$\alpha_2$,\dots}}%
\begin{tabular}{lll}%
参数 & 默认值 & 含义\\
\midrule
\TAline{(pt1=$\alpha_1$,pt2=$\alpha_2$,\dots)}{无}{每个点的权重}
\bottomrule
\end{tabular}

\medskip
注意:至少需要两个已知点,才能实现计算。
\end{NewMacroBox}

% \subsubsection{Using \tkzcname{tkzDefBarycentricPoint} with two points}
\subsubsection{用\tkzcname{tkzDefBarycentricPoint}命令计算2个点的重心}

% In the following example, we obtain the barycentre of points $A$ and $B$ with
% coefficients $1$ and $2$, in other words:
下面的代码中,通过系数\enquote{$1$}和\enquote{$2$}得到了点$A$和点$B$的重心:

\[
  \overrightarrow{AI}= \frac{2}{3}\overrightarrow{AB}
\]

\begin{tkzexample}[latex=7cm,small]
\begin{tikzpicture}
  \tkzDefPoint(2,3){A}
  \tkzDefShiftPointCoord[2,3](30:4){B}
  \tkzDefBarycentricPoint(A=1,B=2)
  \tkzGetPoint{I}
  \tkzDrawPoints(A,B,I)
  \tkzDrawLine(A,B)
  \tkzLabelPoints(A,B,I)
\end{tikzpicture}
\end{tkzexample}

% \subsubsection{Using \tkzcname{tkzDefBarycentricPoint} with three points}
\subsubsection{用\tkzcname{tkzDefBarycentricPoint}命令计算3个点的重心}

% This time $M$ is simply the centre of gravity of the triangle. For reasons of
% simplification and homogeneity, there is also \tkzcname{tkzCentroid}.
下面的代码中,$M$是三角形的重心。

注意,为简化操作,该宏包还提供了另外一个用于直接计算三角形重心的\tkzcname{tkzCentroid}命令。

\begin{tkzexample}[latex=7cm,small]
\begin{tikzpicture}[scale=.8]
  \tkzDefPoint(2,1){A}
  \tkzDefPoint(5,3){B}
  \tkzDefPoint(0,6){C}
  \tkzDefBarycentricPoint(A=1,B=1,C=1)
  \tkzGetPoint{M}
  \tkzDefMidPoint(A,B)  \tkzGetPoint{C'}
  \tkzDefMidPoint(A,C)  \tkzGetPoint{B'}
  \tkzDefMidPoint(C,B)  \tkzGetPoint{A'}
  \tkzDrawPolygon(A,B,C)
  \tkzDrawPoints(A',B',C')
  \tkzDrawPoints(A,B,C,M)
  \tkzDrawLines[add=0 and 1](A,M B,M C,M)
  \tkzLabelPoint(M){$M$}
  \tkzAutoLabelPoints[center=M](A,B,C)
  \tkzAutoLabelPoints[center=M,above right](A',B',C')
\end{tikzpicture}
\end{tkzexample}

% \subsection{Internal Similitude Center}
\subsection{相似中心}

% The centres of the two homotheties in which two circles correspond are called
% external and internal centres of similitude.
两个圆对应外部和内部相似中心。

\begin{tkzexample}[latex=6cm,small]
\begin{tikzpicture}[scale=.75,rotate=-30]
  \tkzDefPoint(0,0){O}
  \tkzDefPoint(4,-5){A}
  % 内部相似中心
  \tkzDefIntSimilitudeCenter(O,3)(A,1)
  \tkzGetPoint{I}
  % 外部相似中心
  \tkzExtSimilitudeCenter(O,3)(A,1)
  \tkzGetPoint{J}
  \tkzDefTangent[from with R = I](O,3 cm)
  \tkzGetPoints{D}{E}
  \tkzDefTangent[from with R = I](A,1 cm)
  \tkzGetPoints{D'}{E'}
  \tkzDefTangent[from with R = J](O,3 cm)
  \tkzGetPoints{F}{G}
  \tkzDefTangent[from with R = J](A,1 cm)
  \tkzGetPoints{F'}{G'}
  \tkzDrawCircle[R,fill=red!50,opacity=.3](O,3 cm)
  \tkzDrawCircle[R,fill=blue!50,opacity=.3](A,1 cm)
  \tkzDrawSegments[add = .5 and .5,color=red](D,D' E,E')
  \tkzDrawSegments[add= 0 and 0.25,color=blue](J,F J,G)
  \tkzDrawPoints(O,A,I,J,D,E,F,G,D',E',F',G')
  \tkzLabelPoints[font=\scriptsize](O,A,I,J,D,E,F,G,D',
                                    E',F',G')
\end{tikzpicture}
\end{tkzexample}

\end{document}
\endinput
